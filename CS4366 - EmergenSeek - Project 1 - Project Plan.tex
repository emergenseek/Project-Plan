\documentclass[10pt]{report}
\usepackage{amsmath,amssymb}
\usepackage{hyperref}

\title{CS 4366: Senior Capstone Project \\ Dr. Sunho Lim \\ Project \#1 - Project Plan - Project Report \\ EmergenSeek}
\author{Suhas Bacchu \ Derek Fritz \ Kevon Manahan \ Annie Vo \ Simon Woldemichael}

\begin{document}

\maketitle

\begin{abstract}
In this report, we propose the development of a mobile application which will provide users with multiuse, centralized emergency information. This application will provide friends and family members with priority connections in times of emergency or crisis
\end{abstract}
\section{Problem Definition and Motivation}
Many people look to travel as a means of discovering new experiences and enjoying brief respites in the middle of their busy lives. Outside of leisure, many companies encourage their employees to travel frequently in the form of consulting roles and teams spread across multiple branches. Many universities, likewise, are promoting studying abroad in order for their students to gain a broader view of the world. In 2017, a total of 2.25 billion person-trips were made domestically and another 76.9 million internationally \cite{one}. The sheer quantity of people traveling is enormous, and the number continues to grow every passing year as more options are made available.

Road travel makes up a large portion of these trips. The United States is one of the busiest countries in terms of road traffic with nearly 264 million vehicles registered and 218 million drivers holding a valid driving license. The level of traffic is one of the reasons leading to more traffic accidents. Road trips represented some 22 percent of vacations taken by United States travelers in 2015, jumping to 39 percent the following year, according to MMGY Global’s 2017-18 Portrait of American Travelers. Seeking convenience and adventure (while avoiding airport security, baggage fees, and the hassle of flying with young children or pets), more travelers are hitting the road to explore unfamiliar places.

With these growing numbers, however, the strictness of regulations continues to impose greater restrictions on travelers. The logistics of planning a trip grow more complex by the day, and this can result in a large amount of stress for potential travelers. Additionally, many people possess concerns regarding safety while traveling to unfamiliar destinations. Daily news stories about violence in large cities and health concerns in developing countries have instilled a general apprehension towards travel. The stress of travel compounded with these very real concerns may discourage many individuals from experiencing the joys of travel.

Herein lies our motivation for proposing an all-in-one, travel-friendly emergency service locator. EmergenSeek aims to alleviate some of the worries that travelers may possess, providing peace-of-mind for travelers and their families regardless of where they may be. By keeping selected family and friends updated on the traveler’s location and safety, informing the user of nearby health service options, and allowing for instantaneous assistance in the form of an SOS button, we hope that EmergenSeek will instill people with a sense of safety even in unfamiliar locations.


\begin{thebibliography}{9}
\bibitem{one}
\url{https://www.ustravel.org/system/files/media_root/document/Research_Fact-Sheet_US-Travel-and-Tourism-Overview.pdf}
\bibitem{two}
\url{https://www.statista.com/statistics/191653/number-of-licensed-drivers-in-the-us-since-1988/}
\end{thebibliography}
\end{document}

